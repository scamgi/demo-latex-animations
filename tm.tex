\documentclass[border=5pt]{standalone}
\usepackage{tikz}
\usepackage{amssymb}
\usepackage{animate} % Pacchetto per le animazioni

\usetikzlibrary{positioning, chains, shapes.geometric}

% Comando per disegnare un singolo frame dell'animazione
% #1: Contenuto del nastro (es. {a,b,a,\Box})
% #2: Posizione della testina (es. tape-1)
% #3: Stato corrente (es. q_0)
\newcommand{\drawtape}[3]{
    \begin{tikzpicture}[
        tape_cell/.style={rectangle, draw, minimum size=1cm, font=\Large},
        tape_head/.style={isosceles triangle, draw, shape border rotate=90, fill=red!30}
    ]
        \begin{scope}[start chain=tape going right, node distance=-0.5pt]
            \foreach \content in #1 {
                \node[on chain, tape_cell] {$\content$};
            }
        \end{scope}
        
        \node[above=0.5cm of #2] (head_label) {$#3$};
        \node[tape_head, above=0.1cm of #2] (head_pointer) {};
    \end{tikzpicture}
}

\begin{document}

% Inizia l'ambiente di animazione
% controls: aggiunge i pulsanti play/pausa
% loop: ripete l'animazione
% 1: frame rate (1 frame al secondo)
\begin{animateinline}[controls,loop]{1}
    
    % Frame 0: Stato iniziale
    \drawtape{{a, b, a, \Box, \Box}}{tape-1}{q_0}
    \newframe

    % Frame 1: Legge 'a', scrive 'b', sposta a destra
    \drawtape{{b, b, a, \Box, \Box}}{tape-2}{q_0}
    \newframe

    % Frame 2: Legge 'b', scrive 'a', sposta a destra
    \drawtape{{b, a, a, \Box, \Box}}{tape-3}{q_0}
    \newframe

    % Frame 3: Legge 'a', scrive 'b', sposta a destra
    \drawtape{{b, a, b, \Box, \Box}}{tape-4}{q_0}
    \newframe

    % Frame 4: Legge 'Box', si ferma, va allo stato finale
    \drawtape{{b, a, b, \Box, \Box}}{tape-4}{q_f}

\end{animateinline}

\end{document}