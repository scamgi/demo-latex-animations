\documentclass{article}
\usepackage[utf8]{inputenc}

\usepackage{chessboard}

\title{Partita di Scacchi in LaTeX}
\author{Il tuo Nome}
\date{\today}

\begin{document}

\maketitle

\section{Posizione Personalizzata}

Per mostrare una posizione specifica, puoi usare il comando \verb|\fenboard{<codice_FEN>}|. Il codice FEN descrive la posizione dei pezzi, di chi è il tratto, le possibilità di arrocco, ecc.

\bigskip

\begin{center}
  % Esempio: Posizione dopo 1. e4 c5 2. Nf3 d6 3. d4 cxd4 4. Nxd4
  \fenboard{rnbqkb1r/pp2pp1p/3p1np1/8/3NP3/8/PPP2PPP/RNBQKB1R w KQkq - 0 5}
\end{center}

\bigskip

Puoi anche aggiungere delle mosse sulla scacchiera.

\begin{center}
    % Stessa posizione, ma con le frecce per mostrare le mosse
    \chessboard[
      setfen=rnbqkb1r/pp2pp1p/3p1np1/8/3NP3/8/PPP2PPP/RNBQKB1R w KQkq - 0 5,
      showmoves,
      moveid=5w,
      pgfstyle=color,
      color=blue,
      arrow=g1-f3,
      markfields={d4,f6}
    ]
\end{center}

\end{document}
